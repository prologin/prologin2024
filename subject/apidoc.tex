% This file was generated by stechec2-generator. DO NOT EDIT.

\noindent \begin{tabular}{lp{11cm}}
\textbf{Constante:} & LARGEUR\_MAX \\
\textbf{Valeur:} & 100 \\
\textbf{Description:} & Largeur maximale de la carte \\
\end{tabular}
\vspace{0.2cm} \\

\noindent \begin{tabular}{lp{11cm}}
\textbf{Constante:} & HAUTEUR\_MAX \\
\textbf{Valeur:} & 100 \\
\textbf{Description:} & Hauteur maximale de la carte \\
\end{tabular}
\vspace{0.2cm} \\

\noindent \begin{tabular}{lp{11cm}}
\textbf{Constante:} & LARGEUR\_MIN \\
\textbf{Valeur:} & 10 \\
\textbf{Description:} & Largeur minimale de la carte \\
\end{tabular}
\vspace{0.2cm} \\

\noindent \begin{tabular}{lp{11cm}}
\textbf{Constante:} & HAUTEUR\_MIN \\
\textbf{Valeur:} & 10 \\
\textbf{Description:} & Hauteur minimale de la carte \\
\end{tabular}
\vspace{0.2cm} \\

\noindent \begin{tabular}{lp{11cm}}
\textbf{Constante:} & NB\_TOURS \\
\textbf{Valeur:} & 400 \\
\textbf{Description:} & Nombre de tours à jouer avant la fin de la partie \\
\end{tabular}
\vspace{0.2cm} \\

\noindent \begin{tabular}{lp{11cm}}
\textbf{Constante:} & GAINS\_MAX \\
\textbf{Valeur:} & 100 \\
\textbf{Description:} & Gains maximum qu'apporte une ile \\
\end{tabular}
\vspace{0.2cm} \\

\noindent \begin{tabular}{lp{11cm}}
\textbf{Constante:} & GAINS\_MIN \\
\textbf{Valeur:} & -100 \\
\textbf{Description:} & Gains minimum qu'apporte une ile \\
\end{tabular}
\vspace{0.2cm} \\

\noindent \begin{tabular}{lp{11cm}}
\textbf{Constante:} & TOUR\_POINTS\_ACTION \\
\textbf{Valeur:} & 2 \\
\textbf{Description:} & Points d'action au début d'un tour \\
\end{tabular}
\vspace{0.2cm} \\

\noindent \begin{tabular}{lp{11cm}}
\textbf{Constante:} & COUT\_ROTATION\_ENNEMI \\
\textbf{Valeur:} & 2 \\
\textbf{Description:} & Coût de rotation d'une case en lien avec une île ennemi \\
\end{tabular}
\vspace{0.2cm} \\

\noindent \begin{tabular}{lp{11cm}}
\textbf{Constante:} & COUT\_ROTATION\_STANDARD \\
\textbf{Valeur:} & 1 \\
\textbf{Description:} & Coût de rotation d'une case qui n'est pas en lien avec une île ennemi \\
\end{tabular}
\vspace{0.2cm} \\

\noindent \begin{tabular}{lp{11cm}}
\textbf{Constante:} & MULTIPLICATEUR\_DERNIER\_TOUR \\
\textbf{Valeur:} & 42 \\
\textbf{Description:} & Multiplicateur de score du dernier tour \\
\end{tabular}
\vspace{0.2cm} \\

\noindent \begin{tabular}{lp{11cm}}
\textbf{Constante:} & LA\_CONSTANTE\_K \\
\textbf{Valeur:} & -41 \\
\textbf{Description:} & k est une constante (relou) \\
\end{tabular}
\vspace{0.2cm} \\


\functitle{erreur} \\
\noindent
\begin{tabular}[t]{@{\extracolsep{0pt}}>{\bfseries}lp{10cm}}
Description~: & Erreurs possibles après avoir effectué une action \\
Valeurs~: &
\small
\begin{tabular}[t]{@{\extracolsep{0pt}}lp{7cm}}
    \textsl{OK}~: & L'action a été effectuée avec succès \\
    \textsl{HORS\_TOUR}~: & Vous ne pouvez pas faire d'action en dehors de votre tour \\
    \textsl{CASE\_BLOQUEE}~: & La case est bloquée par un aigle \\
    \textsl{POSITION\_INVALIDE}~: & La position fournie est invalide \\
    \textsl{DESTINATION\_INVALIDE}~: & La position d'arrivée est invalide \\
    \textsl{PLUS\_DE\_PA}~: & Vous n'avez plus de points d'action \\
    \textsl{AIGLE\_INVALIDE}~: & L'identifiant de l'aigle est invalide \\
    \textsl{ROTATION\_VILLAGE}~: & Vous essayez de tourner un village \\
\end{tabular} \\
\end{tabular}

\functitle{type\_case} \\
\noindent
\begin{tabular}[t]{@{\extracolsep{0pt}}>{\bfseries}lp{10cm}}
Description~: & Contenu topographique d'une case \\
Valeurs~: &
\small
\begin{tabular}[t]{@{\extracolsep{0pt}}lp{7cm}}
    \textsl{VILLAGE}~: & Village \\
    \textsl{NORD\_EST}~: & Case dont le coin manquant est au nord est \\
    \textsl{NORD\_OUEST}~: & Case dont le coin manquant est au nord ouest \\
    \textsl{SUD\_OUEST}~: & Case dont le coin manquant est au sud ouest \\
    \textsl{SUD\_EST}~: & Case dont le coin manquant est au sud est \\
    \textsl{CASE\_INVALIDE}~: & Case invalide \\
\end{tabular} \\
\end{tabular}

\functitle{drakkar\_debug} \\
\noindent
\begin{tabular}[t]{@{\extracolsep{0pt}}>{\bfseries}lp{10cm}}
Description~: & Type de drakkar de debug \\
Valeurs~: &
\small
\begin{tabular}[t]{@{\extracolsep{0pt}}lp{7cm}}
    \textsl{PAS\_DE\_DRAKKAR}~: & Aucun drakkar, enlève le drakkar présent \\
    \textsl{DRAKKAR\_BLEU}~: & Drakkar bleu \\
    \textsl{DRAKKAR\_JAUNE}~: & Drakkar jaune \\
    \textsl{DRAKKAR\_ROUGE}~: & Drakkar rouge \\
\end{tabular} \\
\end{tabular}

\functitle{type\_action} \\
\noindent
\begin{tabular}[t]{@{\extracolsep{0pt}}>{\bfseries}lp{10cm}}
Description~: & Types d'actions \\
Valeurs~: &
\small
\begin{tabular}[t]{@{\extracolsep{0pt}}lp{7cm}}
    \textsl{ACTION\_TOURNER\_CASE}~: & Tourne une case, action ``tourner\_case`` \\
    \textsl{ACTION\_ACTIVER\_AIGLE}~: & Active l'effet d'un aigle, action ``activer\_aigle`` \\
    \textsl{ACTION\_DEPLACER\_AIGLE}~: & Déplace un aigle appartenant à l'utilisateur, action ``deplacer\_aigle`` \\
\end{tabular} \\
\end{tabular}

\functitle{effet\_aigle} \\
\noindent
\begin{tabular}[t]{@{\extracolsep{0pt}}>{\bfseries}lp{10cm}}
Description~: & Effet de l'aigle \\
Valeurs~: &
\small
\begin{tabular}[t]{@{\extracolsep{0pt}}lp{7cm}}
    \textsl{EFFET\_METEORE}~: & Fait tomber un météore qui tourne les cases \\
    \textsl{EFFET\_VIE}~: & Donne des points actions \\
    \textsl{EFFET\_MORT}~: & Effraye les aigles d'un emplacement \\
    \textsl{EFFET\_FEU}~: & Multiplie les gains d'une île \\
    \textsl{EFFET\_GEL}~: & Bloque les rotations de cases \\
\end{tabular} \\
\end{tabular}



\functitle{position}
\begin{lst-c++}
struct position {
    int colonne;
    int ligne;
};
\end{lst-c++}
\noindent
\begin{tabular}[t]{@{\extracolsep{0pt}}>{\bfseries}lp{10cm}}
Description~: & Position dans la carte, donnée par deux coordonnées \\
Champs~: &
\small
\begin{tabular}[t]{@{\extracolsep{0pt}}lp{7cm}}
    \textsl{colonne}~: & Abscisse \\
    \textsl{ligne}~: & Ordonnée \\
\end{tabular} \\
\end{tabular}

\functitle{dimension}
\begin{lst-c++}
struct dimension {
    int largeur;
    int hauteur;
};
\end{lst-c++}
\noindent
\begin{tabular}[t]{@{\extracolsep{0pt}}>{\bfseries}lp{10cm}}
Description~: & Dimensions de la carte \\
Champs~: &
\small
\begin{tabular}[t]{@{\extracolsep{0pt}}lp{7cm}}
    \textsl{largeur}~: & Largeur de la carte \\
    \textsl{hauteur}~: & Hauteur de la carte \\
\end{tabular} \\
\end{tabular}

\functitle{aigle}
\begin{lst-c++}
struct aigle {
    int identifiant;
    int joueur;
    position pos;
    effet_aigle effet;
    int puissance;
    int tour_eclosion;
};
\end{lst-c++}
\noindent
\begin{tabular}[t]{@{\extracolsep{0pt}}>{\bfseries}lp{10cm}}
Description~: & Aigle \\
Champs~: &
\small
\begin{tabular}[t]{@{\extracolsep{0pt}}lp{7cm}}
    \textsl{identifiant}~: & Identifiant de l'aigle \\
    \textsl{joueur}~: & Identifiant du joueur auquel appartient l'aigle, -1 si n'appartient à aucun des joueurs \\
    \textsl{pos}~: & Position de l'aigle \\
    \textsl{effet}~: & Effet de l'aigle \\
    \textsl{puissance}~: & Valeur de la puissance de l'aigle \\
    \textsl{tour\_eclosion}~: & Tour d'éclosion de l'oeuf \\
\end{tabular} \\
\end{tabular}

\functitle{etat\_case}
\begin{lst-c++}
struct etat_case {
    type_case contenu;
    int gains;
    position pos_case;
};
\end{lst-c++}
\noindent
\begin{tabular}[t]{@{\extracolsep{0pt}}>{\bfseries}lp{10cm}}
Description~: & Description complète d'une case \\
Champs~: &
\small
\begin{tabular}[t]{@{\extracolsep{0pt}}lp{7cm}}
    \textsl{contenu}~: & Contenu topographique de la case \\
    \textsl{gains}~: & Gains dans le coin sud-est de la case \\
    \textsl{pos\_case}~: & Position de la case \\
\end{tabular} \\
\end{tabular}

\functitle{action\_hist}
\begin{lst-c++}
struct action_hist {
    type_action action_type;
    position debut;
    position fin;
    int identifiant_aigle;
};
\end{lst-c++}
\noindent
\begin{tabular}[t]{@{\extracolsep{0pt}}>{\bfseries}lp{10cm}}
Description~: & Action représentée dans l'historique \\
Champs~: &
\small
\begin{tabular}[t]{@{\extracolsep{0pt}}lp{7cm}}
    \textsl{action\_type}~: & Type de l'action \\
    \textsl{debut}~: & Position de début de déplacement ou position de la case tournée \\
    \textsl{fin}~: & Position de fin \\
    \textsl{identifiant\_aigle}~: & Identifiant de l'aigle utilisé \\
\end{tabular} \\
\end{tabular}



\begin{minipage}{\linewidth}
\functitle{tourner\_case}
\begin{lst-c++}
erreur tourner_case(position pos)
\end{lst-c++}
\noindent
\begin{tabular}[t]{@{\extracolsep{0pt}}>{\bfseries}lp{10cm}}
Description~: & Rotation d'un quart de tour d'une case dans le sens trigonométrique (anti-horaire) \\
Paramètres~: &
\begin{tabular}[t]{@{\extracolsep{0pt}}ll}
    \textsl{pos}~: & Position de la case \\
  \end{tabular} \\
\end{tabular} \\[0.3cm]
\end{minipage}

\begin{minipage}{\linewidth}
\functitle{activer\_aigle}
\begin{lst-c++}
erreur activer_aigle(int id)
\end{lst-c++}
\noindent
\begin{tabular}[t]{@{\extracolsep{0pt}}>{\bfseries}lp{10cm}}
Description~: & Activer l'effet d'un aigle \\
Paramètres~: &
\begin{tabular}[t]{@{\extracolsep{0pt}}ll}
    \textsl{id}~: & Identifiant de l'aigle \\
  \end{tabular} \\
\end{tabular} \\[0.3cm]
\end{minipage}

\begin{minipage}{\linewidth}
\functitle{deplacer\_aigle}
\begin{lst-c++}
erreur deplacer_aigle(int id, position destination)
\end{lst-c++}
\noindent
\begin{tabular}[t]{@{\extracolsep{0pt}}>{\bfseries}lp{10cm}}
Description~: & Déplace un aigle \\
Paramètres~: &
\begin{tabular}[t]{@{\extracolsep{0pt}}ll}
    \textsl{id}~: & Identifiant de l'aigle \\
    \textsl{destination}~: & Position où l'aigle sera déplacé \\
  \end{tabular} \\
\end{tabular} \\[0.3cm]
\end{minipage}

\begin{minipage}{\linewidth}
\functitle{dimensions\_carte}
\begin{lst-c++}
dimension dimensions_carte()
\end{lst-c++}
\noindent
\begin{tabular}[t]{@{\extracolsep{0pt}}>{\bfseries}lp{10cm}}
Description~: & Renvoie les dimensions hauteur largeur de la carte \\
\end{tabular} \\[0.3cm]
\end{minipage}

\begin{minipage}{\linewidth}
\functitle{info\_case}
\begin{lst-c++}
etat_case info_case(position pos)
\end{lst-c++}
\noindent
\begin{tabular}[t]{@{\extracolsep{0pt}}>{\bfseries}lp{10cm}}
Description~: & Renvoie les informations concernant une case \\
Paramètres~: &
\begin{tabular}[t]{@{\extracolsep{0pt}}ll}
    \textsl{pos}~: & Position de la case \\
  \end{tabular} \\
\end{tabular} \\[0.3cm]
\end{minipage}

\begin{minipage}{\linewidth}
\functitle{info\_aigles}
\begin{lst-c++}
aigle array info_aigles()
\end{lst-c++}
\noindent
\begin{tabular}[t]{@{\extracolsep{0pt}}>{\bfseries}lp{10cm}}
Description~: & Renvoie la liste d'aigles \\
\end{tabular} \\[0.3cm]
\end{minipage}

\begin{minipage}{\linewidth}
\functitle{liste\_villages}
\begin{lst-c++}
position array liste_villages(int joueur)
\end{lst-c++}
\noindent
\begin{tabular}[t]{@{\extracolsep{0pt}}>{\bfseries}lp{10cm}}
Description~: & Renvoie la liste des villages. Identifiant -1 pour les villages libres. \\
Paramètres~: &
\begin{tabular}[t]{@{\extracolsep{0pt}}ll}
    \textsl{joueur}~: & Identifiant du joueur \\
  \end{tabular} \\
\end{tabular} \\[0.3cm]
\end{minipage}

\begin{minipage}{\linewidth}
\functitle{points\_action}
\begin{lst-c++}
int points_action(int joueur)
\end{lst-c++}
\noindent
\begin{tabular}[t]{@{\extracolsep{0pt}}>{\bfseries}lp{10cm}}
Description~: & Renvoie le nombre de points d'action restant. Renvoie -1 si le joueur est invalide. \\
Paramètres~: &
\begin{tabular}[t]{@{\extracolsep{0pt}}ll}
    \textsl{joueur}~: & Identifiant du joueur \\
  \end{tabular} \\
\end{tabular} \\[0.3cm]
\end{minipage}

\begin{minipage}{\linewidth}
\functitle{score}
\begin{lst-c++}
int score(int joueur)
\end{lst-c++}
\noindent
\begin{tabular}[t]{@{\extracolsep{0pt}}>{\bfseries}lp{10cm}}
Description~: & Renvoie le score d'un joueur. Renvoie -1 si le joueur est invalide. \\
Paramètres~: &
\begin{tabular}[t]{@{\extracolsep{0pt}}ll}
    \textsl{joueur}~: & Identifiant du joueur \\
  \end{tabular} \\
\end{tabular} \\[0.3cm]
\end{minipage}

\begin{minipage}{\linewidth}
\functitle{debug\_poser\_drakkar}
\begin{lst-c++}
erreur debug_poser_drakkar(position pos, drakkar_debug drakkar)
\end{lst-c++}
\noindent
\begin{tabular}[t]{@{\extracolsep{0pt}}>{\bfseries}lp{10cm}}
Description~: & Pose un drakkar de debug sur la case indiquée \\
Paramètres~: &
\begin{tabular}[t]{@{\extracolsep{0pt}}ll}
    \textsl{pos}~: & Case où poser le drakkar \\
    \textsl{drakkar}~: & Type du drakkar \\
  \end{tabular} \\
\end{tabular} \\[0.3cm]
\end{minipage}

\begin{minipage}{\linewidth}
\functitle{historique}
\begin{lst-c++}
action_hist array historique()
\end{lst-c++}
\noindent
\begin{tabular}[t]{@{\extracolsep{0pt}}>{\bfseries}lp{10cm}}
Description~: & Renvoie la liste des actions effectuées par l'adversaire durant son tour, dans l'ordre chronologique. Les actions de débug n'apparaissent pas dans cette liste. \\
\end{tabular} \\[0.3cm]
\end{minipage}

\begin{minipage}{\linewidth}
\functitle{recuperer\_territoire}
\begin{lst-c++}
position array recuperer_territoire(int joueur)
\end{lst-c++}
\noindent
\begin{tabular}[t]{@{\extracolsep{0pt}}>{\bfseries}lp{10cm}}
Description~: & Renvoie une liste des positions du territoire d'un joueur. \\
Paramètres~: &
\begin{tabular}[t]{@{\extracolsep{0pt}}ll}
    \textsl{joueur}~: & Identifiant du joueur \\
  \end{tabular} \\
\end{tabular} \\[0.3cm]
\end{minipage}

\begin{minipage}{\linewidth}
\functitle{case\_dans\_rayon}
\begin{lst-c++}
bool case_dans_rayon(int id, position pos)
\end{lst-c++}
\noindent
\begin{tabular}[t]{@{\extracolsep{0pt}}>{\bfseries}lp{10cm}}
Description~: & Renvoie vrai si la case est dans le rayon de l'aigle. Si l'aigle est invalide, renvoie faux. \\
Paramètres~: &
\begin{tabular}[t]{@{\extracolsep{0pt}}ll}
    \textsl{id}~: & Identifiant de l'aigle \\
    \textsl{pos}~: & Position de la case \\
  \end{tabular} \\
\end{tabular} \\[0.3cm]
\end{minipage}

\begin{minipage}{\linewidth}
\functitle{moi}
\begin{lst-c++}
int moi()
\end{lst-c++}
\noindent
\begin{tabular}[t]{@{\extracolsep{0pt}}>{\bfseries}lp{10cm}}
Description~: & Renvoie votre numéro de joueur. \\
\end{tabular} \\[0.3cm]
\end{minipage}

\begin{minipage}{\linewidth}
\functitle{adversaire}
\begin{lst-c++}
int adversaire()
\end{lst-c++}
\noindent
\begin{tabular}[t]{@{\extracolsep{0pt}}>{\bfseries}lp{10cm}}
Description~: & Renvoie le numéro du joueur adverse. \\
\end{tabular} \\[0.3cm]
\end{minipage}

\begin{minipage}{\linewidth}
\functitle{annuler}
\begin{lst-c++}
bool annuler()
\end{lst-c++}
\noindent
\begin{tabular}[t]{@{\extracolsep{0pt}}>{\bfseries}lp{10cm}}
Description~: & Annule la dernière action. Renvoie faux quand il n'y a pas d'action à annuler ce tour-ci. \\
\end{tabular} \\[0.3cm]
\end{minipage}

\begin{minipage}{\linewidth}
\functitle{tour\_actuel}
\begin{lst-c++}
int tour_actuel()
\end{lst-c++}
\noindent
\begin{tabular}[t]{@{\extracolsep{0pt}}>{\bfseries}lp{10cm}}
Description~: & Retourne le numéro du tour actuel. \\
\end{tabular} \\[0.3cm]
\end{minipage}

\begin{minipage}{\linewidth}
\functitle{afficher\_erreur}
\begin{lst-c++}
void afficher_erreur(erreur v)
\end{lst-c++}
\noindent
\begin{tabular}[t]{@{\extracolsep{0pt}}>{\bfseries}lp{10cm}}
Description~: & Affiche le contenu d'une valeur de type erreur \\
Paramètres~: &
\begin{tabular}[t]{@{\extracolsep{0pt}}ll}
    \textsl{v}~: & The value to display \\
  \end{tabular} \\
\end{tabular} \\[0.3cm]
\end{minipage}

\begin{minipage}{\linewidth}
\functitle{afficher\_type\_case}
\begin{lst-c++}
void afficher_type_case(type_case v)
\end{lst-c++}
\noindent
\begin{tabular}[t]{@{\extracolsep{0pt}}>{\bfseries}lp{10cm}}
Description~: & Affiche le contenu d'une valeur de type type\_case \\
Paramètres~: &
\begin{tabular}[t]{@{\extracolsep{0pt}}ll}
    \textsl{v}~: & The value to display \\
  \end{tabular} \\
\end{tabular} \\[0.3cm]
\end{minipage}

\begin{minipage}{\linewidth}
\functitle{afficher\_drakkar\_debug}
\begin{lst-c++}
void afficher_drakkar_debug(drakkar_debug v)
\end{lst-c++}
\noindent
\begin{tabular}[t]{@{\extracolsep{0pt}}>{\bfseries}lp{10cm}}
Description~: & Affiche le contenu d'une valeur de type drakkar\_debug \\
Paramètres~: &
\begin{tabular}[t]{@{\extracolsep{0pt}}ll}
    \textsl{v}~: & The value to display \\
  \end{tabular} \\
\end{tabular} \\[0.3cm]
\end{minipage}

\begin{minipage}{\linewidth}
\functitle{afficher\_type\_action}
\begin{lst-c++}
void afficher_type_action(type_action v)
\end{lst-c++}
\noindent
\begin{tabular}[t]{@{\extracolsep{0pt}}>{\bfseries}lp{10cm}}
Description~: & Affiche le contenu d'une valeur de type type\_action \\
Paramètres~: &
\begin{tabular}[t]{@{\extracolsep{0pt}}ll}
    \textsl{v}~: & The value to display \\
  \end{tabular} \\
\end{tabular} \\[0.3cm]
\end{minipage}

\begin{minipage}{\linewidth}
\functitle{afficher\_effet\_aigle}
\begin{lst-c++}
void afficher_effet_aigle(effet_aigle v)
\end{lst-c++}
\noindent
\begin{tabular}[t]{@{\extracolsep{0pt}}>{\bfseries}lp{10cm}}
Description~: & Affiche le contenu d'une valeur de type effet\_aigle \\
Paramètres~: &
\begin{tabular}[t]{@{\extracolsep{0pt}}ll}
effe    \textsl{v}~: & The value to display \\
  \end{tabular} \\
\end{tabular} \\[0.3cm]
\end{minipage}

\begin{minipage}{\linewidth}
\functitle{afficher\_position}
\begin{lst-c++}
void afficher_position(position v)
\end{lst-c++}
\noindent
\begin{tabular}[t]{@{\extracolsep{0pt}}>{\bfseries}lp{10cm}}
Description~: & Affiche le contenu d'une valeur de type position \\
Paramètres~: &
\begin{tabular}[t]{@{\extracolsep{0pt}}ll}
    \textsl{v}~: & The value to display \\
  \end{tabular} \\
\end{tabular} \\[0.3cm]
\end{minipage}

\begin{minipage}{\linewidth}
\functitle{afficher\_dimension}
\begin{lst-c++}
void afficher_dimension(dimension v)
\end{lst-c++}
\noindent
\begin{tabular}[t]{@{\extracolsep{0pt}}>{\bfseries}lp{10cm}}
Description~: & Affiche le contenu d'une valeur de type dimension \\
Paramètres~: &
\begin{tabular}[t]{@{\extracolsep{0pt}}ll}
    \textsl{v}~: & The value to display \\
  \end{tabular} \\
\end{tabular} \\[0.3cm]
\end{minipage}

\begin{minipage}{\linewidth}
\functitle{afficher\_aigle}
\begin{lst-c++}
void afficher_aigle(aigle v)
\end{lst-c++}
\noindent
\begin{tabular}[t]{@{\extracolsep{0pt}}>{\bfseries}lp{10cm}}
Description~: & Affiche le contenu d'une valeur de type aigle \\
Paramètres~: &
\begin{tabular}[t]{@{\extracolsep{0pt}}ll}
    \textsl{v}~: & The value to display \\
  \end{tabular} \\
\end{tabular} \\[0.3cm]
\end{minipage}

\begin{minipage}{\linewidth}
\functitle{afficher\_etat\_case}
\begin{lst-c++}
void afficher_etat_case(etat_case v)
\end{lst-c++}
\noindent
\begin{tabular}[t]{@{\extracolsep{0pt}}>{\bfseries}lp{10cm}}
Description~: & Affiche le contenu d'une valeur de type etat\_case \\
Paramètres~: &
\begin{tabular}[t]{@{\extracolsep{0pt}}ll}
    \textsl{v}~: & The value to display \\
  \end{tabular} \\
\end{tabular} \\[0.3cm]
\end{minipage}

\begin{minipage}{\linewidth}
\functitle{afficher\_action\_hist}
\begin{lst-c++}
void afficher_action_hist(action_hist v)
\end{lst-c++}
\noindent
\begin{tabular}[t]{@{\extracolsep{0pt}}>{\bfseries}lp{10cm}}
Description~: & Affiche le contenu d'une valeur de type action\_hist \\
Paramètres~: &
\begin{tabular}[t]{@{\extracolsep{0pt}}ll}
    \textsl{v}~: & The value to display \\
  \end{tabular} \\
\end{tabular} \\[0.3cm]
\end{minipage}

