\subsection{C}

\begin{itemize}
\item{Les booléens sont représentés par le type \texttt{bool}, défini par le
      standard du C99, et que l'on retrouve dans le header \texttt{stdbool.h};}
\item{Les fonctions prenant des tableaux en paramètres et retournant des
      tableaux utilisent à la place de ces tableaux une structure
      \texttt{type\_array}, où \texttt{type} est le type des données dans le
      tableau. Ces structures contiennent deux éléments : les données,
      \texttt{type* items}, et la taille, \texttt{size\_t length}. Dans tous les
      cas, la libération des données est laissée au soin du candidat ;}
\item{Tout le reste est comme indiqué dans le sujet.}
\end{itemize}

\subsection{C++}

\begin{itemize}
\item{Les tableaux sont représentés par des \texttt{std::vector<type>} ;}
\item{Le reste est identique au sujet.}
\end{itemize}

\subsection{C\#}

\begin{itemize}
\item{Les fonctions à utiliser sont des méthodes statiques de la classe
      \texttt{Api}.
      Ainsi, pour utiliser la fonction \texttt{Foo}, il faut faire
      \texttt{Api.Foo} ;}
\item{Les noms des fonctions, structures et énumérations sont en
      \texttt{CamelCase}. Ainsi, une fonction nommée \texttt{foo\_bar} dans
      le sujet s'appellera \texttt{FooBar} en C\#.}
\end{itemize}

\subsection{Haskell}

\begin{itemize}
\item{L'API est fournie par le module \texttt{Api}.}
\item{Les énumérations sont représentées par des types sommes, les structures
      par des records. Seule la première lettre des noms de types et de
      constructeurs est en majuscule. Le nom du constructeur d'une structure
      est son nom de type.}
\item{La commande \texttt{make doc} permet de générer la documentation dans le
      fichier \texttt{doc/index.html} pour votre code ainsi que pour l'API.}
\item{Pour pouvoir conserver des valeurs entre différents appels à vos fonctions
      à compléter, il faut utiliser des variables mutables :}
\begin{lstlisting}[language=Haskell]
import Data.IORef
import System.IO.Unsafe (unsafePerformIO)

-- La pragma NOINLINE est importante !
-- MonType ne doit pas etre polymorphe !
{-# NOINLINE maVariable #-}
maVariable :: IORef MonType
maVariable = unsafePerformIO (newIORef maValeurInitiale)

fonctionACompleter :: IO ()
fonctionACompleter = do
  maValeur <- readIORef maVariable
  ...
  writeIORef maVariable maValeur'

\end{lstlisting}
\end{itemize}

\subsection{Java}

\begin{itemize}
\item{Les fonctions à utiliser sont des méthodes statiques de la classe
      \texttt{Interface}. Ainsi, pour utiliser la fonction \texttt{foo}, il
      faut faire \texttt{Interface.foo} ;}
\item{Les structures sont représentées par des classes dont tous les attributs
      sont publics.}
\end{itemize}

\subsection{OCaml}

\begin{itemize}
\item{L'API est fournie par le fichier \texttt{api.ml}, qui est \texttt{open}
      par défaut par le fichier à compléter ;}
\item{Les énumérations sont représentées par des types sommes avec des
      constructeurs sans paramètres. Seule la première lettre des noms des
      constructeurs est en majuscule ;}

\item{Les structures sont représentées par des records, sauf pour la structure
      \texttt{position} qui est représentée par un couple \texttt{int * int} ;}
\item{Les tableaux sont représentés par des \texttt{array} Caml classiques.}
\end{itemize}

\subsection{PHP}

\begin{itemize}
\item{Les constantes sont définies via des \texttt{define} et doivent donc être
      utilisées sans les précéder d'un signe dollar ;}
\item{Les énumérations sont définies comme des séries de constantes. Se référer
      à la puce au-dessus ;}
\item{Les structures sont gérées sous forme de tableaux associatifs. Ainsi, une
      structure contenant un champ \texttt{x} et un champ \texttt{y} sera créée
      comme ceci : \texttt{array('x' => 42, 'y' => 1337)}.}
\end{itemize}

\subsection{Python}

\begin{itemize}
\item{L'API est fournie par le module \texttt{api}, dont tout le contenu est
      importé par défaut par le code à compléter ;}
\item{Les énumérations sont représentées par des \texttt{IntEnum} Python, qui
      peuvent être utilisées comme ceci : \texttt{nom\_enum.CHAMP}. ;}
\item{Les structures sont représentées par des \texttt{NamedTuple} Python, dont
      on peut accéder aux champs via la notation pointée habituelle, et
      qui peuvent être créés comme ceci : \texttt{foo(bar=42, x=3)}, sauf pour
      la structure \texttt{position} qui est représentée par un couple (x, y).}
\end{itemize}

\subsection{Rust}

\begin{itemize}
\item{L'API est fournie par le module \texttt{api}, dont tout le contenu est
      importé par défaut par le code à compléter. ;}
\item{Les noms des structures et énumérations sont en \texttt{CamelCase}.
      Ainsi, une structure nommée \texttt{foo\_bar} dans le sujet
      s'appellera \texttt{FooBar} en Rust.}
\item{Les tableaux sont représentés par des \texttt{Vec<T>} et les strings par
      des \texttt{String}. Les fonctions prennent leurs primitives empruntées
      \texttt{\&[T]} et \texttt{\&str} en entrée.}
\end{itemize}

\subsection{JavaScript}

\begin{itemize}
\item{L'API est définie sur l'objet global et documentée dans le fichier
      \texttt{api.d.ts}.}
\item{Les structures sont représentées par des objets, les énumérations par des
      constantes sur des variables globales, les tuples par des tableaux.}
\item{L'écriture sur la sortie standard et l'erreur standard s'effectue avec
      \texttt{console.log} et \texttt{console.error} respectivement. Les autres
      méthodes standard de \texttt{console} ne sont pas définies.}
\end{itemize}
